% !TeX root = ../thuthesis-example.tex

\chapter{相关工作}
\label{cha:related-work}

\section{三维医学模型重建}

精准的颅颌面手术规划及其效果预测依赖于对术后软组织变形的精确模拟.
这一过程至关重要, 其基础是一个精细刻画解剖结构细节的面部软组织模型.
传统上, 计算机图形学领域利用边界表示法 (如三角形网格) 或体积表示法 (如四面体网格) 来描述实体物体的外形.
然而, 医学成像技术 (如 CT) 获取的数据通常表现为非结构化的体素数据集, 这类数据格局并不适宜直接运用于手术模拟和效果分析.
因而, 确保数据能够满足后继处理流程的需求, 适当的数据预处理工作显得尤为关键.
预处理步骤包括转换并优化原始数据集, 以便生成适宜用于颅颌面外科模拟计算的输入, 从而支持手术规划的精确度和效果预测的可靠性.

Amberg 等研究者 \cite{ambergOptimalStepNonrigid2007} 对经典的迭代最近点 (Iterative Closest Point, ICP) 算法进行了拓展, 使之适应非刚性配准的需求, 并成功保持了其固有的收敛特性.
此算法的创新之处在于引入了多项正则化策略, 并在减少的刚性权重序列中循环求解, 有效地实现模型向目标几何形状的逐步逼近以及全局和局部变形的精确复原.
Amberg 等人的方法中, 一个关键的创新点是应用的局部仿射正则化技术, 为每个网格顶点分配一个仿射变换, 并在相邻顶点之间施加变换差异的约束.
无论是对哪类数据, 该方法都展现出卓越的处理性能.

在计算机图形学领域, 实体的形状通常通过边界表示 (boundary representation) 来定义, 其中三角形网格是一种广泛采用的表达形式.
然而, 如果采用显式的体积表示方式, 例如四面体网格, 那么在动画制作、物理模拟以及几何处理等任务中, 将更能贴近真实世界并提升精确度.
鉴于此, Hang Si \cite{siTetGenDelaunaybasedQuality2015} 开发了 TetGen 软件, 该软件致力于生成高质量四面体网格, 以便为数值分析和科学计算提供更加优良的基础设施.
TetGen 的设计考虑了种种理论与实践上的挑战, 核心采用基于 Delaunay 算法的原则, 以确保其算法的理论正确性.
经验实验证明, TetGen 可以稳定处理各种复杂的三维几何形状, 并在实际应用中显示出了优秀的处理速度.
在四面体网格生成过程中, 理想的情况是能准确地包含 (即, 完美地嵌入) 所提供的边界模型.
尽管如此, 现实世界中的许多模型往往难以实现此目标, 由于它们包含了大量自相交现象、非流形结构和开放边界等缺陷, 这些都是传统网格细分算法难以应对的问题.
为此, Jacobson 等人 \cite{jacobsonRobustInsideoutsideSegmentation2013} 开发了一种稳健的自动算法, 旨在克服上述所有问题, 并实现对输入模型内部体积的精确离散化.
这一方法仅要求输入的三角形网格基本保持一致的方向性, 通过将缠绕数概念拓展到任意三角形上, 定义了一个基于输入模型的划分函数.
当输入为封闭的三角形网格时, 该函数能实现模型的理想分割;而在面临带有多种缺陷的模型时, 它也能呈现出卓越的性能.
划分函数进一步指导了约束 Delaunay 分割 (Constrained Delaunay Tessellation, CDT) 的过程, 确保分割结果满足边界的要求.

在一项最新研究中, Zhang Xiaoyan 等研究者 \cite{zhangEFacetemplateMethodEfficiently2016} 提出一种创新的半自动化技术 --- eFace-Template, 用以高效且精确地构造特定患者的面部软组织三维模型.
此技术的创新之处在于采用了一个基于解剖学细节的模板, 并对其体积形态进行了调整.
这一处理过程结合了标记驱动的混合形状变形技术和稠密表面配准方法, 并应用了薄板样条 (Thin Plate Spline, TPS) 插值法以提升模型的适应性.
这种方法在模型的拟合精度、解剖学准确性以及网格质量方面均达到了较高的标准.

\section{整形手术模拟研究进展}

在整形手术模拟领域, 学界已展开了广泛而深入的探索.
目前的研究主要分为两个主要方向: 首先是数据驱动的方法, 该方法结合先进的数据处理技术以模拟整形手术的多个阶段;其次是基于物理学原理的仿真技术, 通过计算机辅助模型精确预测手术成果.

\subsection{数据驱动的方法}

在计算机视觉领域, 邱泽松等研究人员提出了一种革新性的三维参数化面部模型, 称为 SCULPTOR, 其独特之处在于其高度符合人类骨骼结构的特性 \cite{qiuSCULPTORSkeletonconsistentFace2022a}.
SCULPTOR 模型以 LUCY 数据集为基础, 将面部模型解构为多个维度, 包括形态、姿态和表情, 创造出一种混合模式的表现形式.
该模型以参数驱动方法为核心, 实现了对人类头骨结构、面部几何形态和外观特征的集成建模, 准确捕捉到人脸特征及其背后的骨骼结构.
此外, 模型中引入的特质分量对于模拟由骨骼形态变化引起的面部特征改变尤为关键, 这对于准确预测正颌手术患者术后面貌的变化具有至关重要的作用.

在计算机视觉的领域内, 深度学习技术已经取得了傲人的成就, 尤其是在图像分类、目标检测以及语义分割等任务中表现突出.
近年来, 许多研究者也在探索该技术在手术模拟场景中的潜能.
在此语境下, Ma Lei 等研究者创新性地开发了一种新型的面部模型预测网络, 即 FSC-Net \cite{maSimulationPostoperativeFacial2023}.
该网络的设计宗旨是捕获骨骼形态变化到面部形态变化之间的复杂、非线性关系.
值得称道的是, FSC-Net 基于点变换网络架构, 利用术前与术后成对的数据进行弱监督学习, 以优化网络参数, 这一过程无需严格匹配每一对顶点间的对应关系.
此外, 该网络采用的以距离为指导的形状损失函数, 在对下颌区建模方面注重精确性.
为了维护网格顶点的拓扑一致性并避免产生尖锐或不现实的变形, FSC-Net 在计算局部顶点损失时引入了对顶点位移的约束.
实验评估表明, FSC-Net 在处理速度上相较于传统的有限元方法提高了 15 倍, 同时保持了可媲美传统方法的精度.

在后续的研究中, Ma Lei 等人进一步扩展了之前的工作, 基于 P2P-Net \cite{yinP2PNETBidirectionalPoint2018} 发展了一种新的深度学习框架, 即采用双向点对点卷积网络 (P2P-Conv) \cite{maBidirectionalPredictionFacial2023}.
该框架专门为解决面部与颅骨形态之间的相互转换问题而设计, 以辅助正颌外科手术规划.
P2P-Conv 继承了 P2P-Net 的优势, 并融入了动态点卷积 (PointConv), 成功捕捉了从细节到整体的空间信息.
该框架还实施了多子集数据增强技术, 即通过不同子集的面部和颅骨点数据来增强模型训练.
在推理阶段, 多子集得出的结果会被集成, 以提供一致而准确的密集顶点变换预测.
实验证明, P2P-Conv 在面部与颅骨形态互换预测的准确度上具有显著提升.
尽管如此, 当前方法尚未详尽考虑到易于混淆的个体特异性问题 --- 例如, 一些个体虽然颅骼形状极为相似, 但他们的面部形态可能存在显著差异.

\subsection{物理仿真方法的研究进展}

面部软组织的计算建模技术已广受研究关注, 并取得了深入的探索成果.
根据采用的计算技术种类的区别, 现有的研究主要分为三个类别: 质点-弹簧模型 (Mass-Spring Model, 简称 MSM) 、有限元模型 (Finite Element Model, 简称 FEM) 以及质量-张量模型 (Mass-Tensor Model, 简称 MTM) .

在这三种模型中, 质点-弹簧模型 (MSM) 因其在计算效率方面的显著优越性而被广泛应用于实时面部动画的制作中.
Terzopoulos \cite{terzopoulosPhysicallyBasedFacial1990} 以及 Lee \cite{leeRealisticModelingFacial1995} 是在此领域内的先驱, 他们发展了基于线性肌肉模型的 MSM, 通过这种方法能够高效地模拟面部动画.
同时, Keeve 等研究者 \cite{keeveDeformableModelingFacial1998} 引入了结合棱柱单元的 MSM, 并与有限元方法 (FEM) 在精度与计算成本方面进行了对比分析.

此外, Teschner 等学者 \cite{teschnerDirectComputationNonlinear} 提出了一种改进的多层非线性 MSM, 该模型通过引入静态约束, 能够直接求解面部模拟仿真的平衡状态.
Vicente 及其团队 \cite{vicenteMaxillofacialSurgerySimulation2009} 则专注于应用 MSM 来模拟面部手术后的形态变化.
他们基于六面体元素构成的网格, 设计了一种创新的 MSM.
该模型在理论上与线性有限元模型具有一致性, 团队进而发展了位移尺度调整技术, 以更精准地复现软组织的物理响应行为.

在比较经典的质点弹簧模型 (Mass-Spring Model, MSM) 与有限元模型 (Finite Element Model, FEM) 时, 后者在模拟生物力学特性上呈现了更加卓越的性能, 尽管其计算过程相对较为耗时.
金等研究人员 \cite{kimClinicallyValidatedPrediction2017} 提出了一项创新的三阶段 FEM 方法, 该方法通过模拟组织间的滑动来提升对面部软组织变形预测的准确性.
具体来说, 首阶段依赖 eFace-Template 技术 \cite{zhangEFacetemplateMethodEfficiently2016} 构建了患者的网格模型, 并运用 FEM 模拟了软组织的响应, 同时将手术后骨骼的位移作为固定边界条件应用到模型中.
在第二阶段, 通过节点力约束来模拟牙龈和牙齿之间的滑动作用, 从而显著提升了模拟结果的质量.
第三阶段则重构了软组织与骨骼间的映射关系, 并利用节点间空间约束作为新的边界条件, 为模拟组织滑动效应的仿真质量提供了进一步提升.
研究证实, 这种针对软组织滑动效应的改良 FEM 方法, 在模拟面部整体结构及临床关键区域 --- 尤其是鼻和口唇区域 --- 的精确度上, 与现有其它 FEM 方法相比, 均有显著提升.

为精确预测面部软组织术后的位置变化, Knoops 等研究者 \cite{knoopsNovelSoftTissue2018} 提出了一个创新的概率性有限元分析模型.
该模型将实验设计方法 (Design Of Experiments, DOE) 与迭代优化策略相结合, 成功地生成了一系列包含概率预测区间的三维面部模型.
这些模型主要针对鼻部和上唇区域, 在预测术后位置方面展现了高度的准确性.
相关研究详尽地分析了模型的不准确性和手术计划中的执行不确定性如何影响软组织的预测结果.
此外, 该模型为手术规划提供了考虑预测最小值和最大值的预测区间, 这对于帮助患者充分理解手术可能导致的面部变化十分重要.

质量-张量模型 (Mass-Tensor Model, MTM) 之设计初衷在于平衡计算效率与结果精度.
Cotin 等研究者 \cite{cotinHybridElasticModel2000} 首开先河, 提出建立在 MTM 基础上的混合模型, 旨在精确模拟软组织的局部变形效应.
随后, Picinbono 与合作者 \cite{picinbonoNonlinearAnisotropicElasticity2003} 将此模型扩展至非线性各向异性材料研究之用.
Mollemans 团队 \cite{mollemansPredictingSoftTissue2007} 首次将 MTM 技术运用于颅颌面手术中的软组织模拟, 并从定量和定性两方面通过临床实例验证了该技术的有效性.
Kim等研究者 \cite{kimNewSofttissueSimulation2010} 采用横向各向同性的MTM模型研究面部肌肉行为预测问题, 并提出了一种兼顾生物力学原理和高效仿真的软组织模拟方法.

在这一研究脉络中, Ichim及同事 \cite{ichimPhacePhysicsbasedFace2017} 推出了Phace --- 一项基于物理法则的创新面部动画技术.
该技术与传统的生成方法形成对照, 它通过最小化模拟过程中被动肌肉、主动肌肉与刚体骨骼结构相互作用所产生的非线性势能, 以期更为精准地计算面部表情的变化.
Phace 技术通过整合碰撞与接触处理到仿真模型中, 成功避免了传统动画技术例如混合形状建模所常见的一致性问题.
此技术引入了一种创新的肌肉激活模型, 能够有效再现面部表情的复杂动态.
Phace 的应用潜能极为广阔, 既适用于艺术动画编辑, 也可用于模拟矫形面部手术过程, 乃至模拟面部与外部作用力或物体之间的交互动态.
