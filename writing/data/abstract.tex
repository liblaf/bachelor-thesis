% !TeX root = ../thuthesis-example.tex

% 中英文摘要和关键字

\begin{abstract}
  全球范围内,每年都有许多患者因患有颅颌面畸形而饱受困扰。
  这些畸形不仅可能影响患者的基本生理功能,还会对其心理健康和社会交际产生负面影响。
  正颌手术是治疗颅颌面畸形的重要手段,通过手术可以显著改善患者的生理功能和面部外观。
  术前决策过程中,术后外观预测是患者和医生关注的核心问题,对手术方案的选择至关重要。

  传统的术后外观预测方法主要依赖于医生的主观经验,这种做法往往缺乏客观性和可度量性,且因个体差异较大,预测结果可能不够精确。
  为克服这些限制,本研究提出了一种基于物理模拟的正颌手术术后外观预测新策略。
  通过结合高精度的骨骼和面部重建算法以及生物力学仿真模型,本研究旨在提供更为客观、精确的术后外观预测方法。

  具体来说,本研究实现了两个核心步骤:
  1. 实现了一种高精度且自动化的骨骼和面部重建算法。
  该算法结合了刚性和非刚性配准方法,通过处理患者头部的 CT 影像数据,能够生成高分辨率和一致拓扑的三维模型。
  此方法不仅具有高度的准确性,还显著提升了自动化水平,减少了人工干预所带来的误差。
  2. 利用符合生物力学原理的高效质点-张量模型,对手术后软组织的形态变化进行仿真。
  该模型考虑了软组织的生物力学特性,通过模拟手术过程中骨骼移动对软组织的牵拉和压迫作用,能够精确预测手术后的面部形态变化。
  % 仿真过程中,采用并行计算技术,显著提高了计算效率,确保在短时间内获得高质量的预测结果。

  实验结果显示,本研究提出的预测策略能够有效地预测正颌手术的术后外观,具备较高的精确度和计算效率。
  与传统方法相比,该策略在客观性和准确性方面具有显著优势,为正颌手术的术后外观预测提供了可行的方法,并有潜力成为临床实践中有效的辅助决策工具。
  未来,本研究还将进一步优化模型,结合更多的生物力学数据和临床案例,以提升预测模型的准确性和实用性。

  % 关键词用“英文逗号”分隔,输出时会自动处理为正确的分隔符
  \thusetup{
    keywords = {生物力学建模, 正颌手术模拟, 有限元模型},
  }
\end{abstract}

\begin{abstract*}
  Globally, many patients suffer from craniofacial deformities each year, impacting not only their basic physiological functionalities but also their mental health and social interactions.
  Orthognathic surgery is a crucial method for treating these deformities, significantly enhancing both physiological functionalities and facial appearance.
  In the pre-operative decision-making process, predicting post-operative appearance is a core concern for both patients and doctors and is essential for selecting the appropriate surgical plan.

  Traditional methods for predicting post-operative appearance primarily rely on the subjective experience of doctors.
  This approach often lacks objectivity and measurability, and due to individual differences, the predictions may not be sufficiently accurate.
  To overcome these limitations, this research proposes a new strategy based on physical simulation for predicting post-operative appearance.
  By combining high-precision skeletal and facial reconstruction algorithms with biomechanical models, this research aims to provide a more objective and accurate method for prediction.

  Specifically, this research achieves two core steps:
  1. A high-precision, automatic algorithm for skeletal and facial reconstruction.
  This algorithm combines rigid and non-rigid registration methods, processes CT imaging data of patients' heads, and generates high-resolution three-dimensional models with consistent topology.
  This method not only achieves high accuracy but also significantly enhances automation, reducing errors caused by manual intervention.
  2. An efficient mass-tensor model that adheres to biomechanical principles is used to simulate the morphological changes of soft tissues post-surgery.
  This model considers the biomechanical properties of soft tissues and can accurately predict post-operative facial morphological changes by simulating the effects of bone movement on soft tissues during surgery.

  The experimental results demonstrate that the proposed prediction strategy can effectively predict the post-operative appearance of orthognathic surgery with high accuracy and computational efficiency.
  Compared to traditional methods, this strategy offers significant advantages in terms of objectivity and accuracy, providing a feasible method for predicting post-operative appearance in orthognathic surgery and the potential to become an effective auxiliary decision-making tool in clinical practice.
  In the future, this research will further optimize the model by incorporating more biomechanical data and clinical cases to enhance the accuracy and practicality of the prediction model.

  % Use comma as separator when inputting
  \thusetup{
    keywords* = {Biomechanical Model, Simulation of Orthognathic Surgery, Finite Element Model},
  }
\end{abstract*}
