% !TeX root = ../thuthesis-example.tex

% 中英文摘要和关键字

\begin{abstract}
  正颌手术是治疗颅颌面畸形的重要手段,而术后外观预测在患者术前决策过程中发挥着至关重要的作用。
  然而,传统的预测方法主要依赖于医生的主观经验,这种做法往往缺乏客观性和可度量性。
  为克服这些限制,本研究提出了一种基于物理模拟的正颌手术术后外观预测新策略。
  具体来说,本研究实施了两个核心步骤:
  \begin{enumerate}
    \item 设计并开发了一种高精度且高效率的骨骼和面部重建算法,显著提升了重建任务的准确性和自动化水平。
    \item 利用符合生物力学原理的高效质点-张量模型,对手术后软组织的形态变化进行仿真。
  \end{enumerate}
  实验结果显示,本研究提出的预测策略能够有效地预测正颌手术的术后外观,具备较高的精确度和计算效率。
  本研究为正颌手术的术后外观预测提供了可行的方法,并有潜力成为临床实践中有效的辅助决策工具。

  % 关键词用“英文逗号”分隔,输出时会自动处理为正确的分隔符
  \thusetup{
    keywords = {生物力学建模, 正颌手术模拟, 有限元模型},
  }
\end{abstract}

\begin{abstract*}
  Orthognathic surgery is a pivotal intervention for correcting cranio-maxillofacial anomalies, with the prediction of post-operative appearance assuming a critical role in the pre-operative planning phase.
  Traditional methods of prediction primarily depend on the subjective experience of clinicians, usually lacking in objectivity and quantifiability.
  Addressing these deficiencies, this study introduces an innovative strategy for predicting facial appearance after orthognathic surgery using physical simulation.
  This approach is composed of two principal components:
  \begin{enumerate}
    \item The development of a high-precision and high-efficiency algorithm for skull and facial reconstruction, which markedly enhances the accuracy and level of automation in the reconstruction process.
    \item The application of an advanced Mass Tensor Model, grounded in biomechanical principles, to simulate the post-surgical morphological alterations of soft tissues.
  \end{enumerate}
  Evaluations indicate that the proposed method reliably predicts post-operative facial appearance with notable precision and efficiency.
  This methodology offers a practical solution for anticipating post-surgical outcomes in orthognathic surgery and holds promise as an effective tool for decision-making in clinical applications.

  % Use comma as separator when inputting
  \thusetup{
    keywords* = {Biomechanical Model, Simulation of Orthognathic Surgery, Finite Element Model},
  }
\end{abstract*}
