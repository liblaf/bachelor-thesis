% !TeX root = ../thuthesis-example.tex

\chapter{质点-张量模型}
\label{cha:mtm}

\section{质点-张量模型}

本研究采用基于质点-张量模型 (Mass Tensor Model, 简称 MTM)\cite{cotinHybridElasticModel2000} 来模拟面部软组织的物理行为。
MTM 既具有与质点-弹簧模型 (Mass Spring Model, 简称 MSM) 的相似的简洁结构表达,又能达到与有限元模型 (Finite Element Method, 简称 FEM) 相近的生物力学相关性。
因此 MTM 不仅拥有了 MSM 的高效与易实现特点,同时准确性也不输 FEM。

MTM 基于 Saint Venant-Kirchhoff 应变能模型,具体的能量密度函数可表达为:
\begin{equation} \label{eq:mtm-energy}
  W(\bm{E}) = \frac{\lambda}{2} (\tr(\bm{E}))^2 + \mu \tr(\bm{E}^2)
\end{equation}
其中 $\lambda, \mu$ 分别是材料的 Lam\`e 系数,$\bm{E}$ 代表 Lagrangian 有限应变张量,其定义如下:
\begin{equation}
  \bm{E} = \frac{1}{2} \pqty{(\grad{u})^T + \grad{u} + (\grad{u})^T \cdot \grad{u}}
\end{equation}
这里,$\bm{u}$ 表示位移矢量场。
在小位移假设的情况下,Lagrangian 应变张量 $\bm{E}$ 可以采用线性近似表达为:
\begin{equation}
  \bm{E} = \frac{1}{2} \pqty{\grad{\bm{u}} + (\grad{\bm{u}})^T}
\end{equation}

\subsection{体积离散化}

在进行数字模拟的过程中,将建模对象进行离散化处理是一种常见且有效的手段,其中四面体网格由于其拓扑的简单性和对复杂几何的较好适应性,成为离散化的理想选择。
考虑一个任意的四面体元素 $T_i$,其在无形变状态下的四个顶点位置分别表示为向量 $\bm{p}_0$, $\bm{p}_1$, $\bm{p}_2$, $\bm{p}_3$。
四面体内部任意一点 $\bm{x}$ 的位移场 $\bm{u}$ 可以通过顶点位移的重心坐标线性插值来确定:
\begin{equation}
  \bm{u}(\bm{x}) = \sum_{j = 0}^3 \alpha_j \bm{u}_j
\end{equation}
其中,$\alpha_j$ 代表点 $\bm{x}$ 关于顶点 $\bm{p}_j$ 的重心坐标,表达式为:
\begin{equation}
  \alpha_j(\bm{x}) = - \frac{\bm{m}_j}{6 V_i} \cdot (\bm{x} - \bm{p}_{j + 1})
\end{equation}
在此处,$V_i$ 指代四面体 $T_i$ 的体积,而向量 $\bm{m}_j$ 的定义如下:
% \begin{equation}
%   \begin{split}
%     \bm{m}_j & =
%     \begin{bmatrix}
%       m_j^x & m_j^y & m_j^z
%     \end{bmatrix}
%     ^T                                                                                                                     \\
%              & = \bm{p}_{j + 1} \cp \bm{p}_{j + 2} + \bm{p}_{j + 2} \cp \bm{p}_{j + 3} + \bm{p}_{j + 3} \cp \bm{p}_{j + 1}
%   \end{split}
% \end{equation}
\begin{equation}
  \bm{m}_j
  = \bm{p}_{j + 1} \cp \bm{p}_{j + 2} + \bm{p}_{j + 2} \cp \bm{p}_{j + 3} + \bm{p}_{j + 3} \cp \bm{p}_{j + 1}
\end{equation}
向量 $\bm{m}_j$ 的几何意义较为明显,它的方向与点 $\bm{p}_{j}$ 相对的三角形面片的外法线相同,模长 $\norm{\bm{m}_j}$ 则等同于该三角形面积的两倍。

\subsection{四面体的应变能}

为了便于模型的实际应用,进一步讨论四面体的应变能是必要的。
依据应变能密度函数 $W$ 表达式,\eqref{eq:mtm-energy} 可以改写为:
\begin{equation}
  \begin{split}
    W
     & = \frac{\lambda}{2} (\tr(E))^2 + \mu \tr(E^2)                                                                                                               \\
     & = \frac{\lambda}{2} \pqty{\tr(\frac{\grad{\bm{u}} + (\grad{\bm{u}})^T}{2})}^2 + \mu \tr(\pqty{\frac{\grad{\bm{u}} + (\grad{\bm{u}})^T}{2}}^2)               \\
     & = \frac{\lambda}{2} (\tr(\grad{\bm{u}}))^2 + \frac{\mu}{4} \tr(\pqty{\grad{\bm{u}} - (\grad{\bm{u}})^T}^2) + \mu \tr((\grad{\bm{u}})^T \vdot \grad{\bm{u}}) \\
     & = \frac{\lambda}{2} (\divergence{\bm{u}})^2 - \frac{\mu}{2} \norm{\curl{\bm{u}}}^2 + \mu \tr((\grad{\bm{u}})^T \vdot \grad{\bm{u}})
  \end{split}
\end{equation}
通过上述方程,我们可以更容易地将四面体的应变能表达为位移场的梯度、散度和旋度的函数。
基于上一小节的讨论,我们进一步可以推导出位移场的梯度 ($\grad{\bm{u}}$)、散度 ($\divergence{\bm{u}}$) 以及旋度 ($\curl{\bm{u}}$) 如下所示:
\begin{align}
  \grad{\bm{u}}
   & = \sum_{j = 0}^3 \grad{\alpha_j} \bm{u}_j
  = \sum_{j = 0}^3 - \frac{\bm{m}_j}{6 V_i} \bm{u}_j       \\
  \divergence{\bm{u}}
   & = \sum_{j = 0}^3 \divergence{\alpha_j} \bm{u}_j
  = \sum_{j = 0}^3 - \frac{\bm{m}_j}{6 V_i} \cdot \bm{u}_j \\
  \curl{\bm{u}}
   & = \sum_{j = 0}^3 \curl{\alpha_j} \bm{u}_j
  = \sum_{j = 0}^3 - \frac{\bm{m}_j}{6 V_i} \cp \bm{u}_j
\end{align}

\subsection{逐顶点受力分析}

为了阐述四面体 $T_i$ 内顶点的受力与位移之间的关系,我们继续以下推导:
\begin{equation}
  \bm{f}_j^{T_i}
  = - V_i \pdv{W}{\bm{u}_j}
  = \sum_{k = 0}^3 \bm{K}_{jk}^{T_i} \bm{u}_k
\end{equation}
其中 $\bm{K}_{jk}^{T_i}$ 代表该四面体的刚度矩阵,具体定义如下:
\begin{equation}
  \bm{K}_{jk}^{T_i} = \frac{1}{36 V_i} \pqty{\lambda_i \bm{m}_k^{T_i} (\bm{m}_j^{T_i})^T + \mu_i \bm{m}_j^{T_i} (\bm{m}_k^{T_i})^T + \mu_i (\bm{m}_j^{T_i} \bm{m}_k^{T_i}) \bm{I}_3}
\end{equation}
此处,$V_i$ 表示四面体 $T_i$ 的体积,$\lambda_i$ 和 $\mu_i$ 分别是该四面体的 Lam\`e 系数,$\bm{I}_3$ 是一个 $3 \times 3$ 的单位矩阵,而 $\bm{m}_j^{T_i}$ 的定义如前文所述。
据此,整体模型关于逐顶点受力与位移的关系可以表述为:
\begin{equation}
  \bm{f}_j
  = \sum_{T_i} \bm{f}_j^{T_i}
  = \sum_{k \in \psi_j} \bm{K}_{jk} \bm{u}_k
\end{equation}
其中我们有
\begin{equation}
  \bm{K}_{jk} = \sum_{T_i \in \Lambda_j} \bm{K}_{jk}^{T_i}
\end{equation}
$\Psi_j$ 代表了与顶点 $j$ 邻接的顶点集,$\Lambda_j$ 则表示囊括顶点 $j$ 的四面体集合。
基于此,MTM 在基于质点和四面体的简洁表达上建立了力与位移的线性关系。

\section{边界条件}

针对本文研究问题,我们需要将以上模型运用于正颌外科手术模拟之中。
为了描述正颌手术对软组织影响的模拟过程,力学模型必须能够反映骨骼变化对软组织的作用。
假设骨骼与相邻软组织间的接合点为刚性连接,即在软组织与骨骼相连处,软组织的位移将与骨骼位移一致。
这些点定义为固定顶点;而那些不与骨骼直接相连的软组织顶点,我们定义为自由顶点。
鉴于手术前后骨骼模型的数据已知,我们可以通过计算骨骼位移来推断固定顶点的位移情况。

我们考虑已知的固定顶点,其位移表示为 $\bm{u}_0$,同时求解剩余自由顶点的位移 $\bm{u}_1$。
为此,我们构建了模型的全局刚度矩阵,并表述了整个系统的力-位移关系如下:
\begin{equation}
  \begin{bmatrix}
    \bm{K}_{00} & \bm{K}_{01} \\
    \bm{K}_{10} & \bm{K}_{11}
  \end{bmatrix}
  \begin{bmatrix}
    \bm{u}_0 \\
    \bm{u}_1
  \end{bmatrix}
  =
  \begin{bmatrix}
    \bm{f}_0 \\
    \bm{f}_1
  \end{bmatrix}
\end{equation}
在静态平衡条件下,考虑到自由顶点没有外力作用,即 $\bm{f}_1 = \bm{0}$,可以推导出以下关系:
\begin{equation}
  \bm{K}_{11} \bm{u}_1 = - \bm{K}_{10} \bm{u}_0
\end{equation}
鉴于矩阵 $\bm{K}_{10}$ 的稀疏性质以及其不保证为正定矩阵,常规直接分解方法对于求解此类线性系统并不适用。
因此,我们采用最小残差迭代法 (MINRES) 进行求解,以确保收敛性和计算效率。
