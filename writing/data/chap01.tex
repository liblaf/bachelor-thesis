% !TeX root = ../thuthesis-example.tex

\chapter{引言}

\section{研究背景与意义}

每年, 在全球范围内, 颅颌面 (Craniomaxillofacial, CMF) 畸形使大量患者在功能与美观两方面饱受折磨.
正颌外科手术的主要目标是矫正中颌和下颌的畸形, 无论这些畸形是后天获得的还是先天性的.
具体而言, 外科医生切割下颌和上颌等颌骨, 将它们切分成数个部分, 接着重新排列这些颌骨段, 以恢复颌骨的正常形态.
在整个手术过程中, 外科医生并不直接操纵面部的软组织, 而是依靠颌骨段的重新定位让软组织发生被动改变.

鉴于颅颌面解剖结构的复杂性以及骨组织间的密切相互作用, 正颌外科手术是一项具有较高挑战性的程序.
手术规划通常在实际手术前展开, 并借助计算机辅助外科模拟 (Computer-Aided Surgical Simulation, CASS) 技术来模拟颌骨切割后的理想重建位置.
该规划过程始于通过计算机断层扫描 (CT) 或锥形束 CT (CBCT) 获得的图像来重构三维 (3D) 面部和骨骼模型.
随后, 通过三维头影测量分析对3D模型进行量化, 以评估颌骨畸形的程度.
完成对3D颌骨模型的虚拟切割后, 医生则将分割得到的颌骨段移动到特定位置, 以实现对颌骨畸形的矫正.
依据此规划, 医生将制作相应的手术导板和固定夹板.

在临床实践中, 正颌外科手术的成效往往通过颌骨功能的恢复和术后面部外观改善来评判.
后者尤其对患者的生活质量影响深远, 因此准确模拟术后的面部外观对于正颌外科手术的整体成功至关重要.
这样的模拟能够为外科医生提供有关术后面部外观变化的直观反馈, 并辅助医生在 CASS 中优化颌骨段的定位, 以便更准确地纠正潜在的残余畸形.
传统上, 这种模拟常常采用基于生物力学的建模方法, 例如有限元分析, 其通常包括以下几个步骤:

\begin{enumerate}
  \item 创立高质量的术前软组织患者特异性网格模型,
  \item 为模型赋予生物力学特性和设置边界条件以模拟构件间的物理作用关系,
  \item 通过迭代和逐步移动骨骼部件, 直至外科医生对术后的面部外观感到满意.

\end{enumerate}

\section{研究问题与挑战}

在计算机辅助外科系统 (CASS) 的助力下, 外科医生得以迈向颌骨切割与颌骨段移位规划的准确与高效境地.
然而, 受限于当前技术, 尚缺乏一种实用的方法让外科医生在术前模拟颌骨重定位后的术后面部外观.
鉴于骨骼与面部软组织间存在复杂的物理相互作用, 进行此类模拟工作可谓充满挑战.
目前医生多寄望于术后面部外观能随颌骨段的重定位而“自然”恢复至理想状态, 遗憾的是, 这种期望通常并不如愿, 而是导致面部畸形的手术后遗症.

在正颌手术模拟的研究领域, 各界学者已开展了众多深入的探讨.
目前, 研究重心主要分为两大路径:一是基于数据驱动的研究方法, 借助先进的数据处理技术来模拟手术流程;二是基于物理学原理的仿真技术, 通过计算机模型预测手术效果.

数据驱动方法的效能极度依赖于大规模数据集以及高质量数据.
然而, 在医疗行业, 鉴于隐私保护的法规要求和现实中可获得的病例数据相对有限等因素的影响, 高质量数据的获取面临极大挑战.
这些因素共同作用, 导致数据驱动方法在医学领域的应用可行性遭遇了明显的限制.

与此同时, 传统的基于物理原理的仿真方法多需消耗巨大的计算资源, 并依赖于制作精良的体积网格.
这不仅导致数据预处理过程耗费巨大, 而且对计算性能的高度依赖, 限制了其在临床中的实际应用.

\section{研究内容}

基于上述问题与挑战, 本文构建正颌手术术后外形预测方法:
\begin{enumerate}
  \item 对配准方法进行改进, 使其适应复杂的颌骨结构, 提高配准精度, 减少对人工标注和调整的依赖.
  \item 使用高效且符合生物力学的 MTM 模型, 以模拟手术后软组织的形变.
  \item \todo{完善研究内容}
\end{enumerate}

\section{论文组织结构}

本文的结构将按照以下顺序展开:

第 \ref{cha:related-work} 章: 相关工作. 介绍当前颅颌面外科计算模拟领域的研究现状.
\todo{完善章节内容}
