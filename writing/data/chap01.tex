\chapter{引言}

\section{研究背景与意义}

全球范围内,每年都有许多患者因患有颅颌面 (Craniomaxillofacial, CMF) 畸形而遭受着生理功能及外观审美方面的困扰。
正颌外科手术的治疗目的在于纠正那些由基因缺陷或是受伤等外在因素导致的上下颌畸形。
在实践中,外科医生通过对患者的下颌骨和上颌骨进行精确切割,将颌骨分割成多个部分,进而重新定位这些骨块以修复颌骨的正常形态。
整个操作过程中,医生不是直接操纵面部的软组织,而是通过重新定位硬组织,也就是骨骼,从而促进相应的软组织随之改变形状。

鉴于颅颌面区域的解剖结构之复杂性及相互嵌插的组织关联,正颌手术无疑带有很高的挑战性。
因此,外科医生往往会在术前进行详尽的规划。
他们利用计算机辅助外科模拟 (Computer-Aided Surgical Simulation, CASS) 技术来模拟和预测颌骨切割及重建之后的组织状态。
手术规划的第一部往往是使用计算机断层扫描 (Computed Tomography, CT) 或锥形束计算机断层扫描 (Cone Beam Computed Tomography, CBCT) 等成像技术所获取到的图像,借此构建三维的面部和骨骼模型。
接着,医生将对这些三维模型进行详细分析,以评估颌骨畸形的严重程度,并在虚拟环境中完成颌骨模型的切割及重定位,从而修正畸形。
基于这一步骤的设计,医生将能够制作出对应的手术导板和用于固定颌骨的夹板。

在临床实践中,正颌外科手术结果的评估标准通常基于颌骨功能的恢复以及术后面部外观的改善。
特别是后者,对患者的生活质量有着深远的影响。
因而,精确模拟术后外观变化对整个手术的成功至关重要。
通过这种模拟,外科医生能够直观地预见到术后的面部变化,并在 CASS 系统中调整骨段位置进行优化,达到更精准地纠正残余畸形的目标。
传统的模拟技术通常采用生物力学建模,例如有限元分析,其主要步骤包括:
\begin{enumerate}
  \item 构建患者术前的高质量软组织模型,
  \item 为模型赋予适当的力学属性,并设置边界条件,以模拟组织间的物理行为,
  \item 医生通过逐步移动骨骼部分,反复进行模拟,迭代直至对预期的术后面部外观满意。
\end{enumerate}

\section{研究问题与挑战}

在 CASS 的帮助下,外科医生能够高效率地以高精度进行颌骨的切割与移位预规划。
然而现有技术存在限制,目前尚未有一种有效手段能使外科医生在手术前预测颌骨重定位后的术后面部外观。
考虑到骨骼与面部软组织间存在的复杂相互作用,执行此类模拟任务显得尤其充满挑战。
理想情况下,随着颌骨段的移动,术后的面部外观能自然地达到理想状态,但现实情况往往难以预测,并且容易引发面部畸形的术后并发症。

在正颌手术模拟的研究领域中,学术界已经开展了广泛研究并深入讨论。
主要研究方向可以划分为两个类别:一方面是基于数据驱动的研究方法,这一类方法通过先进的数据处理技术来模拟手术过程;另一方面是基于物理原理的仿真技术,该技术采用计算机模型来预测手术结果。

数据驱动方法的准确性在很大程度上取决于数据集的规模和质量。
然而,在医疗领域,鉴于严格的隐私保护要求以及数量相对有限的实际可用病例,收集高质量数据成为一项极具挑战性的任务。
这种情况下,数据驱动方法在医疗实践中的应用往往受到很大限制。

同时,传统的基于物理原理的仿真方法往往需要投入大量计算资源,并依赖于高质量的体积网格。
这不仅使得数据预处理过程变得格外复杂,还对计算性能有高要求依赖,从而限制了此类方法在临床实践中的广泛应用。

\section{研究内容}

鉴于上述问题和挑战,本文提出如下正颌手术术后面部预测的方法:
\begin{enumerate}
  \item 开发一种精准且高效的骨骼及面部重建方法,从而提升重建工作的准确性和自动化水平。
  \item 实现高效且符合生物力学原理的质点-张量模型,以模拟手术后软组织的形变。
\end{enumerate}

\section{论文组织结构}

本文的组织结构安排如下:

第 \ref{cha:related-work} 章:相关工作。
在本章中,将详尽介绍颅颌面外科计算模拟领域的研究进展与现状。

第 \ref{cha:reconstruction} 章:三维面部和骨骼模型的重建。
本章详述从 CT 数据中提取并重建三维面部和骨骼模型的技术与方法。

第 \ref{cha:mtm} 章:质点-张量模型。
在此章节中,将介绍质点-张量模型的基础原理及其实现方法,并且详细描述该模型在正颌手术模拟中的应用方法。

第 \ref{cha:results} 章:实验结果。
本章详细展示了所提出方法处理真实数据集得到的实验结果,评估了所提出方法的准确性与计算效率。

第 \ref{cha:concolusion} 章:总结与展望。
此章综述了本研究中提出的方法与所完成的工作,并针对未来的研究方向提出了前瞻性的展望。
特别是,本章将讨论有待进一步探究和改进之处,以促进该解决方案的进一步发展。
