\chapter{引言}

\section{研究背景与意义}

在全球范围内, 每年都有大量患者因颅颌面 (Craniomaxillofacial, CMF) 畸形, 在生理功能与审美外观两方面遭受极大的困扰.
正颌外科手术的根本目标, 在于纠正这些由先天因素或后天原因造成的中颌和下颌畸形.
具体来说, 外科医生会对下颌骨和上颌骨执行精准的切割操作, 将它们分解成若干部分后, 再适当地重新固定这些骨段, 以此复原颌骨的正常解剖形态.
整个手术过程中, 医生并非直接操纵面部软组织, 而是依赖于骨骼部分的再定位, 促使软组织发生相应的被动变形.

鉴于颅颌面区域解剖结构之复杂及骨组织间相互作用的紧密性, 正颌外科手术无疑是极富挑战性的.
为此, 医生通常会在手术之前进行详尽的规划, 利用计算机辅助外科模拟 (Computer-Aided Surgical Simulation, CASS) 技术预测和模拟颌骨切割与重建后的理想状态.
这项规划从通过计算机断层扫描 (CT) 或锥形束 CT (CBCT) 获取的图像开始, 依此重建 3D 面部和骨骼模型.
之后, 医生将运用三维头影测量学对三维模型进行详细的量化分析, 此步骤用于评估颌骨畸形的严重程度.
此后, 在完成 3D 颌骨模型的虚拟切割之后, 医生将把切分后的骨段移至特定的新位置, 以此纠正颌骨畸形.
根据此规划, 将进一步制作相应的手术导板和用于固定的夹板.

在临床实践中, 评价正颌外科手术效果的标准, 通常是颌骨功能的恢复及术后面部外观的改善程度.
后一项尤其对患者的生活质量有着深远的影响, 因此, 对术后面部外观进行精确模拟对整个外科手术的成功至关重要.
借助此类模拟, 外科医生可以获得对术后外观变化的直观了解, 并在 CASS 系统中优化骨段位置, 达到更精确矫正潜在残余畸形的目的.
传统的模拟技术通常采用生物力学建模, 如有限元分析, 其主要步骤包括但不限于:

\begin{enumerate}
  \item 创建患者术前高质量的软组织网格模型,
  \item 为模型赋予合适的生物力学属性, 并设定边界条件来模拟结构间的物理相互作用,
  \item 进行迭代的模拟操作, 通过逐步移动骨骼构件, 直到外科医生对预期的术后面部外观满意.
\end{enumerate}

\section{研究问题与挑战}

在计算机辅助外科系统 (CASS) 的支持下, 外科医生能够以高精度和高效率地进行颌骨切割与移位的预规划.
不过, 现有技术的限制导致目前尚无一种有效的手段允许外科医生在术前模拟颌骨重定位后的术后面部外观.
考虑到骨骼与面部软组织之间的复杂相互作用, 执行此类模拟任务是充满挑战性的.
尽管医生普遍期望术后的面部外观能随同颌骨段的移动而 ``自然地'' 达到理想状态, 但现实情况往往不符合预期, 这也易导致面部畸形的术后并发症.

在正颌手术模拟领域, 学术界已进行了广泛的研究, 并展开深入的讨论.
研究的主要焦点分为两个主要方向: 一为基于数据驱动的研究方法, 通过先进的数据处理技术来模拟手术流程;二为基于物理原理的仿真技术, 采用计算机模型预测手术结果.

数据驱动方法的成效在很大程度上依赖于大规模的数据集和高质量的数据.
但是, 在医疗领域, 考虑到隐私保护法规要求以及实际可用的病例数据相对有限, 高质量数据的收集面临严峻挑战.
这些因素的综合作用使得数据驱动方法在医学中的可行性受到显著限制.

此外, 传统基于物理原理的仿真方法往往要求耗费大量的计算资源, 并需要依赖高质量的体积网格.
这不仅使得数据预处理过程变得异常繁琐, 而且对计算性能的依赖性强, 这限制了这些方法在临床实践中的广泛应用.

\section{研究内容}

鉴于前述所述的问题与挑战, 本论文旨在建立以下的正颌手术术后外形预测方法:
\begin{enumerate}
  \item 开发一种精确而高效的骨骼与面部重建方法, 以提升重建工作的准确性及自动化程度.
  \item 采用高效并贴合生物力学的模态变换模型 (MTM) , 模拟手术后软组织的形变情况.
  \item \todo{待进一步完善本研究内容的描述}
\end{enumerate}

\section{论文组织结构}

本论文的组织结构如下:

第 \ref{cha:related-work} 章: 相关工作.
将详细介绍当前颅颌面外科计算模拟领域的研究进展与现状.

\todo{待完善其余章节内容的概述}
