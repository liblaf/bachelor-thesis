% !TeX root = ../thuthesis-example.tex

\chapter{总结与展望}
\label{cha:concolusion}

本论文深入研究了基于 MTM 的正颌手术术后外形预测。
第 \ref{cha:reconstruction} 章详尽探讨了三维面部和骨骼模型的重建方法,全面阐述了如何基于 CT 图像精确重建面部骨骼和外部皮肤表面的技术流程。
接着,第 \ref{cha:mtm} 章系统解释了 MTM 的原理,并将其应用于手术模拟中,实现了符合生物力学原理的术后外形预测。
第 \ref{cha:results} 章通过在临床数据上的实验,验证了所提出方法的有效性与预测的精确度。

在第 \ref{cha:reconstruction} 章中,我们引入了结合刚性配准及非刚性配准的三维模型重建方法。
该方法能成功地将预构建的模板模型与患者特定数据精准配准。
即使在缺少人工设定的特征点的情形下,凭借为配准算法精心设置的参数,我们的方法仍然能够实现高质量的模板与患者数据间的配准。
这大幅降低了数据预处理阶段人工干预的需要。
配准后的模型经过了进一步的修正,并被细分为四面体网格,这为后续的手术模拟与外形预测打下了坚实的基础。

在第 \ref{cha:mtm} 章中,本文详细阐述了 MTM 的原理。
MTM 基于坚实的生物力学理论,保证了模型的正确性。
它不仅保持了结构的简洁性,而且在保证准确性的同时实现了计算的高效率。
我们利用 MTM 进行了手术模拟,根据预期的术后骨骼变化来设定边界条件,并通过精确的计算得出最终的面部外形。

在第 \ref{cha:concolusion} 章中,通过在现实临床条件下获得的 CT 图像数据集上的实验评估,我们证明了所提方法在预测手术后面部形态方面的准确性。
通过对 10 组临床 CT 数据的实验评估显示,我们的研究成果能够较为精确地预测手术后面部形状。

尽管本文所提出的方法在正颌手术后的外形预测问题上取得了一定的成效,但鉴于正颌手术是涉及多方面因素的复杂过程,研究仍需深入。
首先,目前的模型使用的是均质各向同性材料模型,而人面部实际上由多种力学特性各异的组织构成,这一差异可能会影响预测模型的精确度。
其次,当前模型仅考虑了骨骼的变化,未能模拟手术对软组织的影响。
另外,模型还未充分考虑了骨骼与软组织之间的非固连连接,这些连接在实际中可能包含滑动、粘结等复杂行为。
未来工作需致力于细化和完善模型,以更精确地复现手术对各种组织影响的生物力学行为,从而更全面地模拟手术后的实际效果。
